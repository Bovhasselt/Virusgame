\documentclass[12pt]{article}

\usepackage{hyperref}
\usepackage{minted}
\usepackage{tcolorbox}

\title{C++ Programmeermethoden Assignment5}
\author{Bas Terwijn}
\date{\today}

\begin{document}
\maketitle

\section{Introduction}
Assignment5 replaces the C++ Programmeermethoden exam because of
ongoing online teaching due to Corona virus measures and will
determine your grade for this course. In this assignment you are asked
to modify and extend the VirusGame
(\url{https://bitbucket.org/bterwijn/virusgame}) game.

\section{Tasks}
You are asked to do the following tasks but first install and run
VirusGame on your computer using the installation instructions
provided. Then read the documentation and code to get somewhat
familiar with it.

\subsection{Polymorphism}
Currently in VirusGame.cpp all units are stored in a static array:

\begin{tcolorbox}
\begin{minted}{c++}
Virus units[max_nr_units];
\end{minted}
\end{tcolorbox}

But we want to be able to add other classes as units besides only the
'Virus' class. In addition we want to handle the 'player' object as
just another unit so the code get simpler. Therefore change the
'units' array so that units of different classes can be added to it
using polymophism.

\subsection{Avoid Memory Leaks}
With polymorphism you often will dynamically allocate memory when you
instantiate objects using the 'new' keyword. Avoid memory leaks by
de-allocating the memory when it is no longer needed.

\subsection{RAII}
With 'Resource Acquisition Is Initialization' (RAII)

\url{https://en.cppreference.com/w/cpp/language/raii}

you can make sure you, and possible other people that use your code,
won't forget to release or de-allocate any resources such as
memory. This is done by putting the code that releases the resource in
a destructor that is automatically called when an object goes out of
scope. Use RAII to release any resources, for example the memory allocated
for units, in your code.

\subsection{STL Containers}
The modern Standard Template Library containers

\url{http://www.cplusplus.com/reference/stl/}

are the preferred data structures to use. Prefer std::vector over a
static array [] as it can grow to arbitrary size, it knows its own
size, doesn't decay to a pointer when passed to a function, and has
only little additional overhead compared to a static array. Therefore
replace any static array in your code (for example: Virus
units[max\_nr\_units];) with a std::vector and prefer stl containers
if you choose to add data structures.

\subsection{STL Algorithms}
The C++ Core Guidelines recommends using the standard library over
'handcrafted code'

\url{https://isocpp.github.io/CppCoreGuidelines/CppCoreGuidelines#Res-lib}.

Therefore use as much as possible the functions defined in the STL
Algorithms Library

\url{https://en.cppreference.com/w/cpp/algorithm}

instead of for example raw for-loops. For a gentle introduction in STL
Algorithms see:

\url{https://www.youtube.com/watch?v=2olsGf6JIkU}

\subsection{Avoid Duplicate Code}
Avoid having duplicate code or said differently don't repeat yourself
(DRY). The Virus::step() function is currently already a duplicate of
Player::step(). Find a good way to avoid that duplication.

\subsection{Your Own Creative Extension}
The VirusGame is not finished. Extend it so it has interesting game
play. Maybe the player has to avoid touching the viruses, or shoot
them, or bump into them to bounce them into an anti-virus. Maybe also
add some special effects like explosions or track marks. The more
creative the better.

\section{Grading}

Polymorphism          1
Memory leaks          1
RAII                  1
STL Containers        1
STL Algorithms        1
Avoid duplicate code  1
Creative extension    4

simple
https://www.youtube.com/watch?v=O50qTuM5OT0

duplicate code
small functions
good names
const

\section{Git repository}
fork bitbucket private repo

\section{Submission}
deadline
canvas

\end{document}