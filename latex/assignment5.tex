\documentclass[12pt]{article}

\usepackage{hyperref}
\usepackage{minted}
\usepackage{tcolorbox}

\title{C++ Programmeermethoden Assignment5}
\author{Bas Terwijn}
\date{\today}

\begin{document}
\maketitle

\section{Introduction}
Assignment5 replaces the 2020 C++ Programmeermethoden exams because of
ongoing online teaching due to Corona virus measures and will
determine your grade for this course. In this assignment you are asked
to modify and extend the VirusGame
(\url{https://bitbucket.org/bterwijn/virusgame}) game.

\section{Tasks}
You are asked to do the following tasks but first install and run
VirusGame on your computer using the installation instructions
provided. Then read the documentation and code to get somewhat
familiar with it.

\subsection{Polymorphism}
Currently in VirusGame.cpp all units are stored in a static array:

\begin{tcolorbox}
\begin{minted}{c++}
Virus units[max_nr_units];
\end{minted}
\end{tcolorbox}

But we want to be able to add other classes as units besides only the
'Virus' class. In addition we want to handle the 'player' object as
just another unit so the code get simpler. Therefore change the
'units' array so that units of different classes can be added to it
using polymorphism.

\subsection{Avoid Memory Leaks}
With polymorphism you often will dynamically allocate memory when you
instantiate objects using the 'new' keyword. Avoid memory leaks by
de-allocating the memory when it is no longer needed.

\subsection{RAII}
With 'Resource Acquisition Is Initialization' (RAII)

\url{https://en.cppreference.com/w/cpp/language/raii}

you can make sure you, and possible other people that later might use
your code, won't forget to release or de-allocate any resources such
as memory. This is done by putting the code that releases the resource
in a destructor that is automatically called when an object goes out
of scope. Use RAII to release any resources in your code, for example
the memory allocated for units.

\subsection{STL Containers}
The modern Standard Template Library containers

\url{http://www.cplusplus.com/reference/stl/}

are the preferred data structures to use. Prefer std::vector over a
static array [] as it can grow to arbitrary size, it knows its own
size, doesn't decay to a pointer when passed to a function, and has
only little additional overhead compared to a static array. Therefore
replace any static array in your code (for example: Virus
units[max\_nr\_units];) with a std::vector and prefer stl containers
if you choose to add other data structures.

\subsection{STL Algorithms}
The C++ Core Guidelines recommends using the standard library over
'handcrafted code'

\url{https://isocpp.github.io/CppCoreGuidelines/CppCoreGuidelines#Res-lib}.

Therefore use as much as possible the functions defined in the STL
Algorithms Library

\url{https://en.cppreference.com/w/cpp/algorithm}

instead of for example using raw for-loops. For a gentle introduction
in STL Algorithms see the '105 STL Algorithms in Less Than an Hour'
talk by Jonathan Boccara:

\url{https://www.youtube.com/watch?v=2olsGf6JIkU}

\subsection{Avoid Duplicate Code}
Avoid having duplicate code or said differently, don't repeat yourself
(DRY). The Virus::step() function is currently already a duplicate of
Player::step(). Find a good way to avoid that duplication.

\subsection{Your Own Creative Extension}
The VirusGame is not yet finished. Extend it so it has interesting game
play. Maybe the player has to avoid touching the viruses, or shoot
them, or bump into them to bounce them into an anti-virus unit. Maybe also
add some special effects like explosions or track marks. The more
creative the better, make it fun.

\section{Grading}

Polymorphism          2
Avoid Memory leaks    1
RAII                  2
STL Containers        1
STL Algorithms        1
Avoid duplicate code  1
Creative extension    2

Points will be deducted if your code is not simple as described by
Kate Gregory in her 'Simplicity: not just for beginners' talk:

\url{https://www.youtube.com/watch?v=O50qTuM5OT0}

\section{Rules}
You are not allowed to share code with other students, if I detect
(manually or with plagiarism checkers) that different submissions have
similar structure I will have to report that to the examination
board. See the UvA 'Fraude en plagiaat regeling' for more details.

If you optionally choose to fork the VirusGame git repository so you
can use git to track and backup your changes, then make sure the
repository is private otherwise you could get accused of plagiarism if
someone copies your code. Bitbucket allows you to have private
repositories if you use your @uva.nl email address for your user.

\section{Submission}
deadline
canvas

\end{document}