\documentclass[12pt]{article}

\usepackage{minted}
\usepackage{tcolorbox}
\usepackage{xcolor}
\usepackage{hyperref}
\usepackage{graphicx}

\newcommand*{\link}[2]{\href{#1}{\color{blue}\textbf{\textit{#2}}}}

\title{C++ Programmeermethoden Assignment5 herkansing}
\author{Bas Terwijn}
\date{\today}

\begin{document}
\maketitle

\section{Introduction}
Assignment5 replaces the 2020 C++ Programmeermethoden exams because of
ongoing online teaching due to Corona virus measures and will
determine your grade for this course. In this assignment you are asked
to modify and extend the
\link{https://bitbucket.org/bterwijn/virusgame} { VirusGame }
software.
     
\section{Tasks}
You are asked to do the following tasks but first install and run
VirusGame on your computer using the installation instructions
provided and read the documentation and code to get somewhat
familiar with it.

\subsection{task1: Polymorphism}
Currently in VirusGame.cpp all units are stored in a static array:

\begin{tcolorbox}
\begin{minted}{c++}
Virus units[max_nr_units];
\end{minted}
\end{tcolorbox}

But we want to be able to add other classes as units besides only the
``Virus'' class. In addition we want to handle the ``player'' object as
just another unit so the code gets simpler. Therefore change the
``units'' array so that units of different types can be added to it
using dynamic polymorphism.

\subsection{task2: Avoid Duplicate Code}
Avoid having duplicate code or expressions or said differently: don't
repeat yourself (DRY). The Virus::step() function is currently already
a duplicate of Player::step(). Find a good way to avoid that and other
duplication.

\subsection{task3: RAII}
With dynamic polymorphism you will often have to dynamically allocate
memory when you instantiate objects. This could be done using the
``new'' keyword. However you will then have to remember to deallocate
the memory using the ``delete'' keyword when it is no longer needed to
avoid memory leaks. Alternatively with
\link{https://en.cppreference.com/w/cpp/language/raii}{Resource
  Acquisition Is Initialization (RAII)} you can make sure you, and
other people that might use your code in the future, won't forget to
release any resource such as memory. This is done by putting the code
that releases the resource in a destructor that is automatically
called when an object goes out of scope. Use RAII in your code to make
sure all resources are released automatically.

\subsection{task4: STL Containers}
The modern \link{http://www.cplusplus.com/reference/stl/}{Standard
  Template Library containers} are the preferred data structures to
use. Prefer std::vector over a static array as it can grow to
arbitrary size, it knows its own size, it doesn't decay to a pointer when
passed to a function, and has only little additional overhead compared
to a static array. Therefore replace any static array in your code
(for example: Virus units[max\_nr\_units]) with a std::vector and
prefer STL containers if you choose to add other data structures.

\subsection{task5: STL Algorithms}
\link{https://isocpp.github.io/CppCoreGuidelines/CppCoreGuidelines\#Res-lib}
{ES.1 of C++ Core Guidelines} recommends using the standard library
over ``handcrafted code''.  Therefore use as much as possible the
functions defined in the
\link{https://en.cppreference.com/w/cpp/algorithm} {STL Algorithms
  Library} instead of for example raw for-loops. For a gentle
introduction to STL Algorithms see the
\link{https://www.youtube.com/watch?v=2olsGf6JIkU} {``105 STL
  Algorithms in Less Than an Hour''} talk by Jonathan Boccara.

\subsection{task6: Game Extension}
Change the game so that it is similar to the classic game Space Invaders in figure \ref{space_invaders_fig}. See the \link{https://www.youtube.com/watch?v=MU4psw3ccUI} {``Space Invaders 1978 - Arcade Gameplay''} youtube video for an example.

\begin{figure}
\centerline{\includegraphics[scale=0.8]{spaceinvaders.png}}
\caption{Space Invaders.}
\label{space_invaders_fig}
\end{figure}

In this game different types of aliens move left and right and slowly
come down to attack the player. The player can move left and right and
can shoot the aliens. The aliens also sometimes shoot back. You don't
necessarily have to make a exact copy of the game or use any graphics
(colored circles or square will do fine) or write any text as long as
you have a somewhat similar game play. Adding more game features will
increase the points you earn.

\subsection{task7: Templates}
Think of a useful way to use C++ templates (also referred
to as: generic programming, static polymorphism) in your code.

\section{Grading}
Your grade will follow from which tasks you complete to a satisfactory
level:
\begin{center}
\begin{tabular}{ |l|c| } 
  \hline
  \textbf{task}               &  \textbf{points}\\ \hline
  task1: Polymorphism         &               1 \\
  task2: Avoid Duplicate Code &               1 \\
  task3: RAII                 &               1 \\
  task4: STL Containers       &               1 \\
  task5: STL Algorithms       &               2 \\
  task6: Game Extension       &               3 \\
  task7: Templates            &               1 \\
  \hline
\end{tabular}
\end{center}

Points will be deducted if your code is not ``simple'' as described by
Kate Gregory in her \link{https://www.youtube.com/watch?v=n0Ak6xtVXno}
{``Simplicity: not just for beginners''} talk, watch it!

\section{Plagiarism}
You are not allowed to share code with other students, if we detect
(manually or with plagiarism checkers) that different submissions have
similar structure we will have to report that to the examination
board (see the UvA ``Fraude en plagiaat regeling'').

Therefore if you optionally choose to fork the VirusGame git
repository so you can use git to track and backup your changes, then
make sure the repository is private otherwise you could get accused of
plagiarism if someone copies your code. Bitbucket allows you to make
repositories private if you use your ``@uva.nl'' email address for
your profile.

If you base parts of your work on code written by others add
references to the source. Code that is not written/designed by you
will generally not earn you many points so don't rely on others too
much.

\section{Submission}
Submit your code as a zip/tar of the whole VirusGame project before
the deadline on June 25 2021 23:59 on Canvas. Remove the compiled
executables and other derivatives that I don't need to compile your
code in order to reduce the size and complexity. If you add other
dependencies (additional libraries) describe them in the README.md so
I can easily install them. Double check that your submission contains
all required files before you submit.

\end{document}